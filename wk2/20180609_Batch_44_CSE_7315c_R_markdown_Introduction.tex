\documentclass[]{article}
\usepackage{lmodern}
\usepackage{amssymb,amsmath}
\usepackage{ifxetex,ifluatex}
\usepackage{fixltx2e} % provides \textsubscript
\ifnum 0\ifxetex 1\fi\ifluatex 1\fi=0 % if pdftex
  \usepackage[T1]{fontenc}
  \usepackage[utf8]{inputenc}
\else % if luatex or xelatex
  \ifxetex
    \usepackage{mathspec}
  \else
    \usepackage{fontspec}
  \fi
  \defaultfontfeatures{Ligatures=TeX,Scale=MatchLowercase}
\fi
% use upquote if available, for straight quotes in verbatim environments
\IfFileExists{upquote.sty}{\usepackage{upquote}}{}
% use microtype if available
\IfFileExists{microtype.sty}{%
\usepackage{microtype}
\UseMicrotypeSet[protrusion]{basicmath} % disable protrusion for tt fonts
}{}
\usepackage[margin=1in]{geometry}
\usepackage{hyperref}
\hypersetup{unicode=true,
            pdftitle={RMarkdown Introduction},
            pdfborder={0 0 0},
            breaklinks=true}
\urlstyle{same}  % don't use monospace font for urls
\usepackage{color}
\usepackage{fancyvrb}
\newcommand{\VerbBar}{|}
\newcommand{\VERB}{\Verb[commandchars=\\\{\}]}
\DefineVerbatimEnvironment{Highlighting}{Verbatim}{commandchars=\\\{\}}
% Add ',fontsize=\small' for more characters per line
\usepackage{framed}
\definecolor{shadecolor}{RGB}{248,248,248}
\newenvironment{Shaded}{\begin{snugshade}}{\end{snugshade}}
\newcommand{\AlertTok}[1]{\textcolor[rgb]{0.94,0.16,0.16}{#1}}
\newcommand{\AnnotationTok}[1]{\textcolor[rgb]{0.56,0.35,0.01}{\textbf{\textit{#1}}}}
\newcommand{\AttributeTok}[1]{\textcolor[rgb]{0.77,0.63,0.00}{#1}}
\newcommand{\BaseNTok}[1]{\textcolor[rgb]{0.00,0.00,0.81}{#1}}
\newcommand{\BuiltInTok}[1]{#1}
\newcommand{\CharTok}[1]{\textcolor[rgb]{0.31,0.60,0.02}{#1}}
\newcommand{\CommentTok}[1]{\textcolor[rgb]{0.56,0.35,0.01}{\textit{#1}}}
\newcommand{\CommentVarTok}[1]{\textcolor[rgb]{0.56,0.35,0.01}{\textbf{\textit{#1}}}}
\newcommand{\ConstantTok}[1]{\textcolor[rgb]{0.00,0.00,0.00}{#1}}
\newcommand{\ControlFlowTok}[1]{\textcolor[rgb]{0.13,0.29,0.53}{\textbf{#1}}}
\newcommand{\DataTypeTok}[1]{\textcolor[rgb]{0.13,0.29,0.53}{#1}}
\newcommand{\DecValTok}[1]{\textcolor[rgb]{0.00,0.00,0.81}{#1}}
\newcommand{\DocumentationTok}[1]{\textcolor[rgb]{0.56,0.35,0.01}{\textbf{\textit{#1}}}}
\newcommand{\ErrorTok}[1]{\textcolor[rgb]{0.64,0.00,0.00}{\textbf{#1}}}
\newcommand{\ExtensionTok}[1]{#1}
\newcommand{\FloatTok}[1]{\textcolor[rgb]{0.00,0.00,0.81}{#1}}
\newcommand{\FunctionTok}[1]{\textcolor[rgb]{0.00,0.00,0.00}{#1}}
\newcommand{\ImportTok}[1]{#1}
\newcommand{\InformationTok}[1]{\textcolor[rgb]{0.56,0.35,0.01}{\textbf{\textit{#1}}}}
\newcommand{\KeywordTok}[1]{\textcolor[rgb]{0.13,0.29,0.53}{\textbf{#1}}}
\newcommand{\NormalTok}[1]{#1}
\newcommand{\OperatorTok}[1]{\textcolor[rgb]{0.81,0.36,0.00}{\textbf{#1}}}
\newcommand{\OtherTok}[1]{\textcolor[rgb]{0.56,0.35,0.01}{#1}}
\newcommand{\PreprocessorTok}[1]{\textcolor[rgb]{0.56,0.35,0.01}{\textit{#1}}}
\newcommand{\RegionMarkerTok}[1]{#1}
\newcommand{\SpecialCharTok}[1]{\textcolor[rgb]{0.00,0.00,0.00}{#1}}
\newcommand{\SpecialStringTok}[1]{\textcolor[rgb]{0.31,0.60,0.02}{#1}}
\newcommand{\StringTok}[1]{\textcolor[rgb]{0.31,0.60,0.02}{#1}}
\newcommand{\VariableTok}[1]{\textcolor[rgb]{0.00,0.00,0.00}{#1}}
\newcommand{\VerbatimStringTok}[1]{\textcolor[rgb]{0.31,0.60,0.02}{#1}}
\newcommand{\WarningTok}[1]{\textcolor[rgb]{0.56,0.35,0.01}{\textbf{\textit{#1}}}}
\usepackage{longtable,booktabs}
\usepackage{graphicx,grffile}
\makeatletter
\def\maxwidth{\ifdim\Gin@nat@width>\linewidth\linewidth\else\Gin@nat@width\fi}
\def\maxheight{\ifdim\Gin@nat@height>\textheight\textheight\else\Gin@nat@height\fi}
\makeatother
% Scale images if necessary, so that they will not overflow the page
% margins by default, and it is still possible to overwrite the defaults
% using explicit options in \includegraphics[width, height, ...]{}
\setkeys{Gin}{width=\maxwidth,height=\maxheight,keepaspectratio}
\IfFileExists{parskip.sty}{%
\usepackage{parskip}
}{% else
\setlength{\parindent}{0pt}
\setlength{\parskip}{6pt plus 2pt minus 1pt}
}
\setlength{\emergencystretch}{3em}  % prevent overfull lines
\providecommand{\tightlist}{%
  \setlength{\itemsep}{0pt}\setlength{\parskip}{0pt}}
\setcounter{secnumdepth}{0}
% Redefines (sub)paragraphs to behave more like sections
\ifx\paragraph\undefined\else
\let\oldparagraph\paragraph
\renewcommand{\paragraph}[1]{\oldparagraph{#1}\mbox{}}
\fi
\ifx\subparagraph\undefined\else
\let\oldsubparagraph\subparagraph
\renewcommand{\subparagraph}[1]{\oldsubparagraph{#1}\mbox{}}
\fi

%%% Use protect on footnotes to avoid problems with footnotes in titles
\let\rmarkdownfootnote\footnote%
\def\footnote{\protect\rmarkdownfootnote}

%%% Change title format to be more compact
\usepackage{titling}

% Create subtitle command for use in maketitle
\newcommand{\subtitle}[1]{
  \posttitle{
    \begin{center}\large#1\end{center}
    }
}

\setlength{\droptitle}{-2em}
  \title{RMarkdown Introduction}
  \pretitle{\vspace{\droptitle}\centering\huge}
  \posttitle{\par}
  \author{}
  \preauthor{}\postauthor{}
  \date{}
  \predate{}\postdate{}


\begin{document}
\maketitle

{
\setcounter{tocdepth}{2}
\tableofcontents
}
\hypertarget{assignment--1}{%
\subsection{Assignment -1}\label{assignment--1}}

\textbf{This is first assignment }

\hypertarget{multiply-13-with-9-and-then-subtract-10-and-add-35}{%
\subsection{Multiply 13 with 9 and then subtract 10 and add
35?}\label{multiply-13-with-9-and-then-subtract-10-and-add-35}}

\hypertarget{divide-this-result-by-42}{%
\section{Divide this result by 42}\label{divide-this-result-by-42}}

\hypertarget{code-chunk}{%
\subsection{Code Chunk:}\label{code-chunk}}

this is a sentence

\begin{Shaded}
\begin{Highlighting}[]
\NormalTok{m<-}\DecValTok{13}\OperatorTok{*}\DecValTok{9-10}\OperatorTok{+}\DecValTok{35}
\KeywordTok{print}\NormalTok{(}\StringTok{' What will be the final output?'}\NormalTok{)}
\end{Highlighting}
\end{Shaded}

\begin{verbatim}
## [1] " What will be the final output?"
\end{verbatim}

\begin{Shaded}
\begin{Highlighting}[]
\NormalTok{m <-}\StringTok{ }\NormalTok{m}\OperatorTok{/}\DecValTok{42}
\KeywordTok{print}\NormalTok{(m)}
\end{Highlighting}
\end{Shaded}

\begin{verbatim}
## [1] 3.380952
\end{verbatim}

\hypertarget{code-chunk-options}{%
\subsection{Code Chunk Options:}\label{code-chunk-options}}

Chunk output can be customized with knitr options, arguments set in the
\{\} of a chunk header{[}reference: 1{]} :

\begin{itemize}
\tightlist
\item
  include = FALSE prevents code and results from appearing in the
  finished file. R Markdown still runs the code in the chunk, and the
  results can be used by other chunks.\\
\item
  echo = FALSE prevents code, but not the results from appearing in the
  finished file. This is a useful way to embed figures.\\
\item
  message = FALSE prevents messages that are generated by code from
  appearing in the finished file.\\
\item
  warning = FALSE prevents warnings that are generated by code from
  appearing in the finished.\\
\item
  fig.cap = ``\ldots{}'' adds a caption to graphical results.
\end{itemize}

\hypertarget{rmarkdown-working}{%
\subsection{Rmarkdown Working:}\label{rmarkdown-working}}

This may sound complicated, but R Markdown makes it extremely simple by
encapsulating all of the above processing into a single render function.
{[}reference: 1{]}

\hypertarget{rmarkdown-basics}{%
\subsection{RMarkdown Basics:}\label{rmarkdown-basics}}

\hypertarget{priting-in-italics}{%
\subsubsection{Priting in italics:}\label{priting-in-italics}}

\emph{Italics}\\
\emph{Italics}

\hypertarget{priting-in-bold}{%
\subsubsection{Priting in bold:}\label{priting-in-bold}}

\textbf{Bold}\\
\textbf{Bold}

\hypertarget{wiritng-sentences}{%
\subsubsection{Wiritng Sentences:}\label{wiritng-sentences}}

This is sentence 1.\\
This is sentence 2.

\hypertarget{block-quotes}{%
\subsubsection{Block Quotes:}\label{block-quotes}}

\begin{quote}
This is a block quote
\end{quote}

\hypertarget{embedding-links}{%
\subsubsection{Embedding links:}\label{embedding-links}}

\url{http://rmarkdown.rstudio.com}

\href{\%5Bhttp://rmarkdown.rstudio.com\%5D}{RMarkdown with R studio}

\hypertarget{unorderd-list}{%
\subsubsection{Unorderd List:}\label{unorderd-list}}

\begin{itemize}
\tightlist
\item
  This is item 1
\item
  This is item 2

  \begin{itemize}
  \tightlist
  \item
    This is sub-item 2-1
  \end{itemize}
\end{itemize}

\hypertarget{orderd-list}{%
\subsubsection{Orderd List:}\label{orderd-list}}

\begin{enumerate}
\def\labelenumi{\arabic{enumi}.}
\tightlist
\item
  This is item 1
\item
  This is item 2

  \begin{itemize}
  \tightlist
  \item
    This is sub-item 2-1
  \end{itemize}
\end{enumerate}

\hypertarget{superscript}{%
\subsubsection{Superscript:}\label{superscript}}

superscript\textsuperscript{2}\\
10\textsuperscript{2}

\hypertarget{subscript}{%
\subsubsection{Subscript:}\label{subscript}}

superscript\textsubscript{2}

\hypertarget{equations}{%
\subsubsection{Equations:}\label{equations}}

\begin{enumerate}
\def\labelenumi{\arabic{enumi}.}
\item
  Mean in words\\
  \[{Mean} = \frac{Summation~of~values}{Total~sample~size}\]
\item
  Mean in mathematical expressions
  \[{\mu} = \frac{\sum_{i=1}^n X_i}{n}\]
\end{enumerate}

\hypertarget{images}{%
\subsubsection{Images:}\label{images}}

\hypertarget{get-images-from-net}{%
\subsubsection{Get images from net}\label{get-images-from-net}}

Standard Deviation equation:
\includegraphics{http://www.statisticshowto.com/wp-content/uploads/2013/11/sample-standard-deviation.jpg}

\hypertarget{some-other-commonly-used-symbols}{%
\paragraph{Some other commonly used
symbols:}\label{some-other-commonly-used-symbols}}

\textbf{Sigma:} \(\sigma\)\\
\textbf{Pi:} \(\pi\)\\
\textbf{Alpha:} \(\alpha\)\\
\textbf{Lambda:} \(\lambda\)\\
\textbf{Greater than or equal to:} \(\ge\)\\
\textbf{Less than or equal to:} \(\le\)\\
\textbf{Plus or minus:} \(\pm\)

\hypertarget{lets-write-some-r-code-now}{%
\subsection{Lets write some R code
now:}\label{lets-write-some-r-code-now}}

\hypertarget{view-the-data}{%
\paragraph{1. View the data}\label{view-the-data}}

\begin{Shaded}
\begin{Highlighting}[]
\NormalTok{volcanic_data =}\StringTok{ }\NormalTok{faithful}
\KeywordTok{head}\NormalTok{(volcanic_data)}
\end{Highlighting}
\end{Shaded}

\begin{verbatim}
##   eruptions waiting
## 1     3.600      79
## 2     1.800      54
## 3     3.333      74
## 4     2.283      62
## 5     4.533      85
## 6     2.883      55
\end{verbatim}

\begin{verbatim}
## Warning: package 'knitr' was built under R version 3.4.3
\end{verbatim}

\begin{longtable}[]{@{}rr@{}}
\caption{Volcanic data table in a neat way}\tabularnewline
\toprule
eruptions & waiting\tabularnewline
\midrule
\endfirsthead
\toprule
eruptions & waiting\tabularnewline
\midrule
\endhead
3.600 & 79\tabularnewline
1.800 & 54\tabularnewline
3.333 & 74\tabularnewline
2.283 & 62\tabularnewline
4.533 & 85\tabularnewline
2.883 & 55\tabularnewline
\bottomrule
\end{longtable}

\hypertarget{let-us-do-a-scatter-plot}{%
\paragraph{2. Let us do a scatter plot}\label{let-us-do-a-scatter-plot}}

\includegraphics{20180609_Batch_44_CSE_7315c_R_markdown_Introduction_files/figure-latex/unnamed-chunk-4-1.pdf}

\hypertarget{what-is-the-summary-of-the-dataset}{%
\paragraph{3. What is the summary of the
dataset}\label{what-is-the-summary-of-the-dataset}}

\begin{Shaded}
\begin{Highlighting}[]
\KeywordTok{summary}\NormalTok{(volcanic_data)}
\end{Highlighting}
\end{Shaded}

\begin{verbatim}
##    eruptions        waiting    
##  Min.   :1.600   Min.   :43.0  
##  1st Qu.:2.163   1st Qu.:58.0  
##  Median :4.000   Median :76.0  
##  Mean   :3.488   Mean   :70.9  
##  3rd Qu.:4.454   3rd Qu.:82.0  
##  Max.   :5.100   Max.   :96.0
\end{verbatim}

\begin{Shaded}
\begin{Highlighting}[]
\DecValTok{2}\OperatorTok{+}\StringTok{ }\DecValTok{4}
\end{Highlighting}
\end{Shaded}

\begin{verbatim}
## [1] 6
\end{verbatim}

\hypertarget{references}{%
\subsection{References:}\label{references}}

\begin{enumerate}
\def\labelenumi{\arabic{enumi}.}
\tightlist
\item
  \url{https://www.rstudio.com/wp-content/uploads/2015/02/rmarkdown-cheatsheet.pdf}
\item
  \url{http://rmarkdown.rstudio.com}
\end{enumerate}

\hypertarget{practice}{%
\subsection{Practice:}\label{practice}}

\begin{enumerate}
\def\labelenumi{\arabic{enumi}.}
\tightlist
\item
  Refer to the lecture notes and write the following equations in
  Rmarkdown:
\end{enumerate}

\begin{itemize}
\tightlist
\item
  mean
\item
  median
\item
  variance
\item
  standard deviation
\item
  IQR
\end{itemize}

\[{\sigma} = \sqrt{\frac{\sum_{i=1}^n (X_i - \bar{X})^2}{n-1}}\]

\begin{enumerate}
\def\labelenumi{\arabic{enumi}.}
\setcounter{enumi}{1}
\tightlist
\item
  Read in the internal dataset mtcars and see the summary of the dataset
  in Rmakrdown. Report the mean of horse power (column hp) as hp\_bar =
  xxx value (write hp\_bar in equation notations)
\end{enumerate}

\begin{Shaded}
\begin{Highlighting}[]
\NormalTok{cars_data =}\StringTok{ }\NormalTok{mtcars}
\KeywordTok{summary}\NormalTok{(mtcars)}
\end{Highlighting}
\end{Shaded}

\begin{verbatim}
##       mpg             cyl             disp             hp       
##  Min.   :10.40   Min.   :4.000   Min.   : 71.1   Min.   : 52.0  
##  1st Qu.:15.43   1st Qu.:4.000   1st Qu.:120.8   1st Qu.: 96.5  
##  Median :19.20   Median :6.000   Median :196.3   Median :123.0  
##  Mean   :20.09   Mean   :6.188   Mean   :230.7   Mean   :146.7  
##  3rd Qu.:22.80   3rd Qu.:8.000   3rd Qu.:326.0   3rd Qu.:180.0  
##  Max.   :33.90   Max.   :8.000   Max.   :472.0   Max.   :335.0  
##       drat             wt             qsec             vs        
##  Min.   :2.760   Min.   :1.513   Min.   :14.50   Min.   :0.0000  
##  1st Qu.:3.080   1st Qu.:2.581   1st Qu.:16.89   1st Qu.:0.0000  
##  Median :3.695   Median :3.325   Median :17.71   Median :0.0000  
##  Mean   :3.597   Mean   :3.217   Mean   :17.85   Mean   :0.4375  
##  3rd Qu.:3.920   3rd Qu.:3.610   3rd Qu.:18.90   3rd Qu.:1.0000  
##  Max.   :4.930   Max.   :5.424   Max.   :22.90   Max.   :1.0000  
##        am              gear            carb      
##  Min.   :0.0000   Min.   :3.000   Min.   :1.000  
##  1st Qu.:0.0000   1st Qu.:3.000   1st Qu.:2.000  
##  Median :0.0000   Median :4.000   Median :2.000  
##  Mean   :0.4062   Mean   :3.688   Mean   :2.812  
##  3rd Qu.:1.0000   3rd Qu.:4.000   3rd Qu.:4.000  
##  Max.   :1.0000   Max.   :5.000   Max.   :8.000
\end{verbatim}

\begin{Shaded}
\begin{Highlighting}[]
\NormalTok{m =}\StringTok{ }\KeywordTok{mean}\NormalTok{(mtcars}\OperatorTok{$}\NormalTok{cyl)}
\end{Highlighting}
\end{Shaded}

The mean of number of cylinders is: \[ \bar{cyl} = 6.1875\]


\end{document}
